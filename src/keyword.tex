\documentclass[main.tex]{subfiles}
\begin{document}

\section{Explain the following C keywords: \texttt{volatile}, \texttt{const}, and \texttt{static}.}

\spoilerline

\subsection{Volatile}
The \textbf{volatile} keyword is used to tell the compiler that a variable's value can be \textit{changed unexpectedly} (the value can be changed by something outside the current scope, such as an interrupt service routine (ISR) or by hardware via memory-mapped input/outputs, or I/O). The compiler should not optimize the variable's access, as it may change at any time.

\subsubsection{Example Volatile Use Case}
Consider a short program below in listing \ref{code:volatile-use-case} where a timer interrupt service routine (ISR) increments a variable \texttt{timer\_counter} every millisecond. 

\lstinputlisting[caption={Volatile Use Case}, label={code:volatile-use-case}]{code/keyword/volatile_ex.c}

\noindent The main loop checks if the \texttt{timer\_counter}'s modulo (remainder) has reached a certain value and then performs an action. If the \texttt{timer\_counter} variable is not declared as \texttt{volatile}, the compiler may optimize the loop and cache the value of \texttt{timer\_counter} after reading it only once as the ISR\_Timer function does not appear to be called, causing the if statement to never be true. Effectively, the compiler assumes that you, the human, has written dead code (\textit{term for code that is unreachable}) and thinks it can optimize it away. By declaring \texttt{timer\_counter} as \texttt{volatile}, the compiler will always read the variable from memory, ensuring the loop works as expected.
\newline
\newline
\textbf{Note:} A major use-case of volatile is to access memory-mapped I/O registers in embedded systems. These registers can change due to external events by the HW rather than code, and the compiler should not optimize the reads/writes to these registers. Memory-mapped I/O registers should be declared as such (ex: \texttt{volatile uint32\_t * const UART\_DR = (uint32\_t *)0x40000000;}) to avoid the aforementioned issues.

\subsection{Const}
The \textbf{const} keyword refers to a variable that is \textbf{read-only} \cite{beningoConst}. The compiler will throw an error if an attempt is made to modify a variable labeled as \texttt{const}. \textbf{Importantly, \texttt{const} does not mean that the value is constant, but rather that the variable cannot be modified by the program}. The distinction is important, as it's possible to have values that are declared as \texttt{const}, but are not constant. 
\newline
\newline
\noindent Const's primary use is to make code more readable and maintainable. By declaring a variable as \texttt{const}, the programmer can signal to others that the variable should not be modified and have the compiler enforce this behaviour. This can help prevent bugs and make the code easier to understand. As a general rule, it's good practice to declare variables as \texttt{const} whenever possible.

\subsubsection{Example Const Use Case}
Consider the code snippet in listing \ref{code:const-use-case} below. The variable \texttt{UART\_RECEIVE\_REGISTER\_READ\_ONLY} is declared as \texttt{const}, meaning that the value stored at the memory address \texttt{0x12345678} cannot be modified. If an attempt is made to modify the value stored at this address, the compiler will throw an error. However, it's important to note that the value stored at the memory address is not constant, but rather the variable \texttt{UART\_RECEIVE\_REGISTER\_READ\_ONLY} is read-only, so the pointed to can change when read, despite being declared as const. In this case, the value stored at the memory address can change due to external events, such as hardware (in this case, UART) writing to the memory address.

\begin{lstlisting}[caption={Example use case of Const}, label={code:const-use-case}]
#include <stdint.h>

volatile const uint32_t* const UART_RECEIVE_REGISTER_READ_ONLY = (uint32_t*)0x12345678;

int main() {
    *UART_RECEIVE_REGISTER_READ_ONLY = 0xDEADBEEF; // Compiler error: assignment of read-only location
    return 0;
}
\end{lstlisting}

\subsection{Static}
The \textbf{static} keyword has different meanings depending on the context in which it is used.
\subsubsection{Static Functions}
When used with functions, the \texttt{static} keyword limits the function's scope to the file in which it is defined (as all functions are implicitly declared as extern without the \texttt{static} qualifier \cite{arm_static}). This means that the function cannot be accessed by other files through linking. This is useful for helper functions that are only used within a single file and should not be exposed to other files, in effect creating a private function. Listing \ref{code:static-function} shows an example of a static function.
\newpage

\begin{lstlisting}[caption={Example of a Static Function}, label={code:static-function}]
static void helper_function() {
    // Function implementation. This function can only be accessed within the file it is defined in.
}
\end{lstlisting}

\subsubsection{Static Declarations}
When used with global variables, the \texttt{static} keyword limits the variable's scope to the file in which it is defined \cite{arm_static}. This means that the variable cannot be accessed by other files through linking. This is useful for creating private global variables that are only accessible within a single file. Ex: declaring \texttt{static uint8\_t counter = 0U;} in a file means \texttt{counter} is accessible only within that file, and no where else. These variables are stored in the data segment, or Block Starting Symbol (BSS) segment if uninitialized, and retain their value throughout the program's execution.\footnote{For more information, see \href{https://cosmic-software.com/faq/faq17.php}{this FAQ entry on static variables}.}


\subsubsection{Static Local Variables}
When used with local variables within functions, the \texttt{static} keyword changes the variable's storage class to static. This means that the variable is stored in the data segment rather than the stack (where temporary data and variables are stored \cite{embedded_com_Static}), and its value is retained between function calls. This is useful when you want a variable to retain its value between function calls. Listing \ref{code:static-local-variable} shows an example of a static local variable.

\lstinputlisting[caption={Static Local Variable}, label={code:static-local-variable}]{code/keyword/main.c}


\end{document}