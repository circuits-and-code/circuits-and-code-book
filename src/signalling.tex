\documentclass[main.tex]{subfiles}
\begin{document}

\section{You have a sensor positioned a meter away from a central controller, what type of signal would you use to transfer the data?}
The sensor is a temperature sensor and readings are needed at a rate of 100 Hz with a precision of 0.1C and a temperature range of -40C to 125C.

\spoilerline

\noindent I2C or UART protocol should be selected in this case. 

\subsection{Digital Signalling}
% What is a digital signal

% Differential vs. Single ended
% Busses vs point to point protocols
% Full-Duplex vs Half-Duplex
% Push Pull vs Open Drain 

The most common digital protocols in embedded systems are:
\begin{itemize} % maybe turn this into a table? show what protocols are what?
    \item PWM (\textit{Pulse Width Modulation})
    \item UART (\textit{Universal Asynchronous Receiver Transmitter}) is a single-ended serial
    \item I2C 
    \item SPI
    \item CAN
    \item Ethernet
    \item USB
\end{itemize}

% Logic Level & Shifting

\subsection{Analog Signalling}
% What is an analog signal

Microcontrollers, the brains of embedded systems, compute digitally so analog signals are often overlooked, however, there are numerous reasons for relying on them including: 
\begin{itemize}
    \item All signals from real sensors begin as "analog" values so for any IC (\texit{integrated circuit}) that outputs a digital value, when you pay for that IC, you're also paying for their onboard ADC.
    \item Analog signals simple and cheap to implement. Microcontrollers often feature ADCs with numerous available input pins and external analog muxes (\textit{multiplexers}) are very cheap. This means it is easy to have a microcontroller sample numerous analog signals on a PCB for a low cost.
    \item Simple signal processing can be performed in hardware before digital computation is required. The most common example of analog signal processing in hardware is first order low pass filter. 
    \item More complex analog circuits are used in applications where higher bandwidth is needed in control loops. Analog feedback loops are very common in power electronics, but the flexibility and easy modification of 
\end{itemize}

% differential vs single ended analog signals, maybe some thermistor circuits 

% Be sure to properly compare analog and digital signals, when would you want to use both of them

\end{document}
