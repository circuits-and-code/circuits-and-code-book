\documentclass[main.tex]{subfiles}
\begin{document}

\section{How would you select an oscilloscope and probe if you wanted to analyze the signal integrity of a 40 mbps SPI signal on a PCB?}

\subsection{Oscilloscopes}
An oscilloscope is a device that plots voltage vs time in a two dimensional graph. It does this by utilizing an ADC (\textit{Analog to Digital Converter}) to discretize a continuous time continuous value signal into quantized samples that can be plotted on a monitor. Consequently the oscilloscope can perform many digital logic operations on the signal offering a plethora of features for engineers. 

\subsection{Triggering}
As the signals we are usually analyzing in electronics are far faster than humans can process, oscilloscopes use the concept of triggering to set a $t=0s$ point from a signal. This allows a periodic high frequency signal to appear stationary on the oscilloscope screen when the trigger is set correctly. 

The most common form of trigger is known as edge triggering. When rising edge trigger mode is used, the oscilloscope begins a displayed image at $t=0s$ every time the sampled signal rises across a voltage threshold. Other trigger types occur, specifically for digital protocols, to trigger on more complex conditions such as a specific series of bytes in a digital protocol. 

Another parameter of the trigger is the mode. In normal mode the oscilloscope will only trigger when a trigger event occurs. In auto mode the oscilloscope will automatically trigger if a trigger event has not occurred for some period of time. The oscilloscope can also be stopped which means no trigger events are allowed meaning the display does not update.  

\subsection{Probes}
The most basic oscilloscope probes have a ground clip and probe hook connected through a coaxial cable to a BNC connector that connects to the oscilloscope. These probes are designed to have a relatively high impedance and affect the circuit under test minimally. 

These basic probes usually feature a slider for 1x and 10x operating modes. Note that more advanced probes may feature different multiplications and simpler probes may be fixed at a single multiplication. These multiplications scale the voltage the probe is receiving before it enters the oscilloscope. At 1x mode the sampled voltage is directly passed to the oscilloscope. At higher multiplications, a voltage divider inside the probe is used to scale down the probed voltage before the oscilloscope. At multiplications lower than 1x the probe is an active probe and contains some sort of amplifier internally, this is very uncommon. Oscilloscopes have a maximum input voltage on their analog to digital converters which makes using 10x or 100x probes sometimes required. Another advantage of using 10x mode instead of 1x mode is higher bandwidth is offered. Oscilloscopes in their channel configuration menus allow you to software compensate for these scaling factors so the oscilloscope display shows the real voltage being probed rather than the voltage being sampled.

Simple probes have a ground connection that is required to have a quality signal. Ground on each scope probe are shorted together and  are shorted to earth ground by the oscilloscope. This is done for safety, but can make differential voltage measurements difficult. Consider attempting to measure the voltage drop across a resistor in which neither end is connected to ground. To do this with single ended probes two probes must be used; one on each end of the resistor, each referenced to the same ground. Then the oscilloscope can mathematically compute the difference between each sampled signal. This gives rise to differential voltage probes which provide isolation. In this scenario a single differential probe can be used to probe the drop across the resistor.

Current clamps are another type of probe that are used to measure current by outputting a voltage which can be connected to an oscilloscope. Oscilloscopes have the mathematical functionality to do this conversion in software. Current clamps work on the principle of magnetic current sensing and have active amplifier in them.

Oscilloscopes feature a sample waveform output, usually a 1 kHz square wave. This is intended for verifying probes are functional and the oscilloscope is setup correctly.

\subsection{Sample Rate}
Digital multi-meters (\textit{DMMs}) also sample voltage with respect to time, however, their sample rate is significantly lower than that of an oscilloscope.
% sample rate implications when selecting a scope

\subsection{Bandwidth}
The bandwidth of a low pass filter is defined by the frequency in which half the power, or $\frac{1}{\sqrt{2}}$ of the voltage, of an input signal passes to the output signal. In the context of wide-band measurements and oscilloscopes, the bandwidth of the circuits is always discussed in this regard though broader definitions of bandwidth are used in analyzing more complex circuitry.

Oscilloscopes and measurement probes each have their own bandwidth specifications. 
% elaborate

\end{document}
