\documentclass[main.tex]{subfiles}
\begin{document}

\section{Write a C function to determine the direction of stack growth on your system.}
\spoilerline

\subsection{Understanding the Stack}
The stack is a region of memory that is used to store local variables and function call information. The stack is a \textit{stack} data structure, meaning that data comes in as though it was last-in, first-out (LIFO). This means that the last item placed on the stack is the first item to be removed. The stack is used to store local variables, function arguments, and the return address of the function\footnote{It's worth noting that the direction of stack growth is usually a strictly academic question - there is little use in only knowing the direction of stack growth.}. This is opposed to \textit{heap memory}, which is allocated \textit{dynamically} and is used to store data that persists beyond the scope of a function call and is managed by the user using \texttt{malloc} and \texttt{free}.

\subsubsection{Stack Frame Example}
When a function is called a new \textit{stack frame} is created, the return address is pushed onto the stack, and when the function returns, the return address is popped off the stack and execution. If you've heard of the terms stack trace, this is where it comes from - looking through the contents of the stack to find the 'path' of execution as well as the contents of the arguments supplied.
\newline
\newnoindentpara Consider the following C snippet in this example, on a system where the stack grows downwards (from higher addresses to lower addresses). 
\lstinputlisting[caption={Stack Example}, label={code:stack-ex}]{code/stack/stack_ex/main.c}

\noindent Generally speaking, the stack might look something like what is found in Table \ref{table:stack_overview} if we paused inside the \texttt{multiply} function. Note the stack grows downwards, hence the decrease in address values. Also note that a new stack frame is created for each function call.

\begin{table}[H]
    \centering
    \begin{tabular}{|c|c|c|c|}
    \hline
    \textbf{Address} & \textbf{Function}            & \textbf{Called From}   & \textbf{Arguments}        \\ \hline
    0x5000           & \texttt{multiply}           & \texttt{add\_and\_multiply} & \texttt{c = 10, d = 20}   \\ \hline
    0x6000           & \texttt{add\_and\_multiply} & \texttt{main}           & \texttt{a = 10, b = 20}   \\ \hline
    0x7000           & \texttt{main}               & \texttt{\_start}        & \texttt{(N/A)}            \\ \hline
    \end{tabular}
    \caption{High-level overview of the stack during program execution in \texttt{multiply}.}
    \label{table:stack_overview}
\end{table}

\subsection{Determining Stack Growth Direction}
Since we know that a new stack frame is created for each function call, we can use this property to determine the direction of stack growth. The steps to determine the direction of stack growth are as follows:
\begin{enumerate}
    \item Create a main function that calls a dummy function. Pass in a pointer to a stack-allocated variable to the dummy function.
    \item In the dummy function, compare the address of the stack-allocated variable to the address of a local variable in the dummy function.
    \item If the address of the stack-allocated variable is less than the address of the local variable, the stack grows downwards. If the address of the stack-allocated variable is greater than the address of the local variable, the stack grows upwards.
\end{enumerate}
\noindent The code snippet in listing \ref{code:stack-growth} demonstrates this concept.

\lstinputlisting[caption={Determining Stack Growth Direction}, label={code:stack-growth}]{code/stack/main.c}
    

\end{document}