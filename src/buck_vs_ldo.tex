\documentclass[main.tex]{subfiles}
\begin{document}

\section{When would you use a Buck Converter or a Low Dropout Regulator?}

\spoilerline

\subsection{Low Dropout Regulator}
\begin{figure}[h!]
    \begin{circuitikz}
        % DRAW: schematic, this is proving to be quite difficult, you'd think chatgpt would be capable but i guess not ....
    \end{circuitikz}
    \caption{Low Dropout Regulator Conceptual Schematic}
\end{figure}

% Efficiency of LDO 
% Conduction and Quiescent Losses
% Applications
% Dropout Voltage

\subsection{Buck Converter}
\begin{figure}[h!]
    \begin{center}
        \begin{circuitikz}
            \draw (0, 4) node[anchor=east] {$V_{in}$} -- (0.5, 4);
            \draw (0.5, 4) to[switch] (2, 4); % TODO : figure out how to make switch bigger so it looks better
            \draw (2, 2) to[diode] (2, 4);
            \draw (2, 4) to[inductor, l=$L$] (4, 4);
            \draw (4, 2) to[C, l=$C$] (4, 4);
            \draw (4, 4) -- (5, 4) node[right] {$V_{out}$};
            \draw (-0.5, 2) -- (5.5, 2);
            \draw (3, 2) node[ground]{};
        \end{circuitikz}
    \end{center}
    % TODO : it adds a very large new line after the figure, figure out how to fix this
    \caption{Asynchronous Buck Converter Conceptual Schematic}
\end{figure}

% Switching and Conduction Losses as a concept
% CCM vs DCM ? 
% Efficiency

\subsection{Comparison}
% When to use either

\subsection{Power Tree}
% Series vs parallel converter tradeoffs
% Power tree design

\end{document}
