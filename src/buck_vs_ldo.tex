\documentclass[main.tex]{subfiles}
\begin{document}

\section{When would you use a Buck Converter or a Low Dropout Regulator?}

\spoilerline

\subsection{Low Dropout Regulator}
LDOs (\textit{Low Dropout Regulators}) are commonly available as ICs (\textit{Integrated Circuits}). LDOs are most commonly used as a voltage source for circuits that are characterized by requiring low voltage low current, and high stability, such as microcontrollers and analog circuitry. They are also ideal in space constrained applications as they can be extremely small. A schematic representation of an LDO is shown in figure \ref{fig:low_dropout_regulator}.

\begin{figure}[H]
    \begin{center}
        \begin{circuitikz}
            \draw (0, 4) node[anchor=east] {$V_{in}$} -- (1, 4); 
            \draw (1.5,4) node[npn, rotate=90, scale=1.5] (npn) {}
                (1,4) to[short] (npn.C)
                (npn.E) to[short] ++(3.5,0) node[right] {$V_{out}$};
            \draw (5.5, 4) to[resistor] (5.5, 2);
            \draw (5.5, 2) to[resistor] (5.5, 0);
            \draw (5.5, 0) node[ground]{};
            \draw (3,1.6) node[op amp, rotate=180, yscale=-1, scale=0.8] (opamp) {}
                (opamp.+) node[right] {$V_{ref}$}
                (opamp.-) to[short] (5.5, 2);
            \draw (opamp.out) -- (1.5, 1.6) -- (npn.B);
            \label{fig:low_dropout_regulator}
        \end{circuitikz}
    \end{center}
    \caption{Low Dropout Regulator Conceptual Schematic}
\end{figure}

\subsubsection{Selection}
When selecting an LDO for a given application, consider optimizing the following parameters in component selection to save board space and cost while ensuring functionality. 
\begin{itemize}
    \item \textbf{Dropout Voltage:} To maintain a stable output voltage, the LDO requires sufficient input voltage. Mathematically, this is expressed as $V_{\text{in}} > V_{\text{out}} + V_{\text{dropout}}$. The $V_{\text{dropout}}$ varies between components but is usually around 0.5 V.
    \item \textbf{Maximum Input Voltage:} The semiconductor pass element inside the transistor is rated to handle a specific maximum input voltage.
    \item \textbf{Maximum Output Current:} The pass element semiconductor has a specific limit to the maximum output current it can provide. Exceeding this limit can lead to overheating or failure. This relates to the maximum power dissipation of the package as there is a limit to the amount of heat that can be dissipated into the environment. 
    \item \textbf{Output Voltage:} Some LDOs feature adjustable output voltages with a feedback pin, while others have fixed output voltages.
\end{itemize}

\subsubsection{Losses}
In an LDO, the pass element transistor (usually implemented as a PMOS FET or NPN BJT) acting similarly to a variable resistor is actively controlled to maintain a fixed output voltage, regardless of changes to input voltage, load current, and temperature. This variable resistance gives rise to \textit{conduction losses} within this type of regulator. Assuming a fixed input voltage and load current, the power losses in an LDO can be calculated as $P = V_{drop} \cdot I_{out} = (V_{in} - V_{out}) \cdot I_{out}$. Note that the power is dissipated as heat in the pass element - this is why LDOs are often equipped with a heat sink, and attention may be required to the thermal limitations of the chosen LDO. \newline

\noindent Another source of losses in an LDO is quiescent losses - the LDO requires a small amount of current to operate, even when the load current $I_{out} = 0$. This is known as the quiescent current, and is usually negligible compared to conduction losses.

\subsection{Buck Converter}
The buck converter is a simple yet fundamental circuit in power electronics. It is a \textit{Switched-Mode Power Supply} (SMPS) that converts a higher input voltage to a lower output voltage. Note that buck converters and associated topics are frequently covered in electrical engineering interviews. An example of a buck converter is shown in figure \ref{fig:buck_converter}.

\begin{figure}[H]
    \begin{center}
        \begin{circuitikz}
            \draw (0, 4) node[anchor=east] {$V_{in}$} -- (0.5, 4);
            \draw (0.5, 4) to[switch] (2, 4); % TODO : figure out how to make switch bigger so it looks better, for some reason using scale breaks it ... 
            \draw (2, 2) to[diode] (2, 4);
            \draw (2, 4) to[inductor, l=$L$] (4, 4);
            \draw (4, 2) to[C, l=$C$] (4, 4);
            \draw (4, 4) -- (5, 4) node[right] {$V_{out}$};
            \draw (-0.5, 2) -- (5.5, 2);
            \draw (3, 2) node[ground]{};
            \label{fig:buck_converter}
        \end{circuitikz}
    \end{center}
    % TODO : it adds a very large new line after the figure, figure out how to fix this
    \caption{Asynchronous Buck Converter Conceptual Schematic}
\end{figure}

\subsubsection{Operation}
When a buck converter is operational, the high side switch turns on during $T_{on}$ and off during $T_{off}$. The duty cycle of a buck converter is defined as $D_{on} = \frac{T_{on}}{T_{on} + T_{off}}$. The switching frequency of a buck converter is given by $f_{SW} = \frac{1}{T_{on} + T_{off}}$. Buck converters generally operate under a roughly constant switching frequency, but modulate their duty cycle with an active control loop to produce a stable output voltage. The current paths for each state of the circuit are depicted below. 

% DRAW : Buck Circuit Current Path HS ON

% DRAW : Buck Circuit Current Path HS OFF

\noindent Buck converters usually operate in CCM (\textit{Continuous Conduction Mode}) meaning there is constantly current flowing forward in the inductor. In CCM, $D_{on} = \frac{V_{out}}{V_{in}} = \frac{T_{on}}{T_{on} + T_{off}}$, describes the duty cycles of an ideal buck converter as shown in the following figure.

\begin{figure}[H]
    \centering
    \includegraphics[width=0.8\textwidth]{generated_images/buck_ccm.png}
    \caption{Buck Converter in Continuous Conduction Mode}
    \label{fig:buck-ccm}
\end{figure}

When $I_{L_{avg}}$ decreases below some threshold, the buck converter circuit enters DCM (\textit{Discontinuous Conduction Mode}) in which $I_L$ reaches $I_L = 0$.

% DRAW : DCM Plotted Current Waveform (I_L and V_sw)

When $I_L \approx 0$ the switch node voltage, $V_{sw}$ begins to oscillate with a peak of $\approx V_{out} \cdot 2$ converging towards $V_{out}$. This is because when $I_L \approx 0$, $V_{sw} = 0$, however, the output node is not at zero, $V_{out} \neq 0$ which means current will begin to flow through the inductor into the parasitic capacitance, $C_{parasitic}$ that exists between $V_{sw}$ and ground. This small resonant current, $I_L$, results in a noticable voltage fluctuation on $V_{sw}$, but not on $V_{out}$ because $C_{out} >> C_{parasitic}$ where $C_{out}$ is the output capacitance of the buck converter. 

\subsubsection{Switching}
The switch depicted in figure \ref{fig:buck_converter} is commonly referred to as a high side switch. In practice, the high side switch is usually implemented with a MOSFET (\textit{Metal-oxide semiconductor field effect transistor}) that features fast switching speeds, low switching losses, low conduction losses, low cost, and low leakage. The gate of this MOSFET requires active control from a feedback loop to maintain a stable output voltage.

\noindent An asynchronous buck converter is depicted in Figure \ref{fig:buck_converter} depicts the low side switch as a diode. The diode is nice to use because it does not require control and is cheaper than a transistor. Diodes, due to their forward voltage drop, have higher conduction losses than MOSFETs. Synchronous buck converters replace this diode with a MOSFET. 

% Explain concept of deadtime and current flowing in the body diode of low side FET during dead time 

\subsubsection{Losses}
In a buck converter there are two major forms of loss: switching losses and conduction losses. Switching losses is power dissipated every time the FETs are switched (transitioned from off to on, or vice-versa) and scale proportionally with switching frequency. Conduction losses are due to parasitic resistance of elements in the converter and consequently scale proportionally to load current. Buck converters have some quiescent current associated with their active control loop, however, these losses are negligible compared to switching and conduction losses and are not usually analyzed. 

For a majority of applications, especially when $I_{load}$ is large and/or $\frac{V_{out}}{V_{in}}$ is very small, buck converters are significantly more efficient than LDOs. Typical buck converter efficiency, $\frac{P_{out}}{P_{in}}$, exceeds 90 percent.  

\subsubsection{Switching Frequency}
Selection of switching frequency is critical in the design of a buck converter. Increasing switching frequency increases switching losses in the converter, but allows the usage of a lower inductance inductors which are usually cheaper and come in smaller packages. Switching frequencies of modern buck converters range from roughly 100 KHz to 5 MHz. 

\subsection{Comparison}
Buck converters are more efficient than LDOs in a majority of applications though an LDO may be optimal in applications where $V_{in} \approx V_{out}$ and/or $I_{out} \approx 0$. 

Buck converters require more physical space than LDOs on a circuit board, mostly because they need an inductor to operate. The need for an inductor and switching circuitry also makes buck converters more expensive than LDOs. 

LDOs produce a more stable output voltage and can have higher control loop bandwidth as they do not have a switching stage nor an output filter as compared to a buck converter. Both converter topologies require input and output capacitance to produce relatively stable output voltages.

\subsubsection{Power Tree}
Embedded system controllers often have relatively high voltage power sources, to minimize current and in turn conduction losses, and are tasked with distributing lower voltage power to endpoints. Usually a single controller board is responsible for driving numerous loads with each having their own power requirements, most notably a roughly fixed voltage. 

Consider a hypothetical controller in an automobile mobile controller tasked with powering a 24V motor and a 5V sensor from a 48V source. A solution could be to convert 48V to 24V for the motor and use another converter to convert 48V to 5V. However, a more efficient approach is to convert 48V to 24V and supplying the 24V to the motor and another converter to produce 5V for the sensor. There are numerous tradeoffs between operating cascaded versus parallel converters in a power tree including: failure modes, low power states, and conversion efficiency. 

\subsection{Follow-ups}
\begin{itemize}
    \item When would you use Diode Emulation Mode and Forced Pulse Width Modulation Mode for a Buck Converter?  % Deeper understanding of inductor current
    \item Why are input capacitors required? What happens if you do not have input capacitors? % Parasitic Inductance & Filtering, oscillations occur on switch node can overvolt low-side FET. 
    \item What are some considerations when selecting output capacitors? % Ripple & Stability
    \item What are key considerations for measuring efficiency of a buck converter? What does the test setup look like? % Efficiency load current dependence, probing voltage directly at input and output but current can be anywehre, etc.
    % So many good follow-ups to this buck converter question that get asked! LDOs are much simpler but yea.
\end{itemize}

\end{document}
