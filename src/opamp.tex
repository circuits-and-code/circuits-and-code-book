\documentclass[main.tex]{subfiles}
\begin{document}

\section{Give the transfer function for the following circuits.}

\noindent In a real interview expect just one of these circuits.
% Numerous circuits are analyzed for the sake of building intuition and the showing first principles approach.

\begin{figure}[h!]
    \begin{center}
        \begin{circuitikz}[american]
            \draw (0,0) node[left]{$V_{in}$} to[short] ++ (1,0)
                node[op amp, noinv input up, anchor=+](opamp){}
                (opamp.-) -- ++(0,-1) coordinate(FB)
                (FB) to[short] (FB -| opamp.out) -- (opamp.out)
                to [short] ++(1,0) node[right]{$V_{out}$};
            \label{fig:unity_amp}
        \end{circuitikz}
        \caption{Circuit A}
    \end{center}
\end{figure}

\begin{figure}[h!]
    \begin{center}
        \begin{circuitikz}[american]
            \draw (0, 0) node[left]{$V_{in}$};
            \draw (0, 0) to[resistor, l=$R_i$] (2, 0);
            \draw (2, 0) node[op amp, anchor=-](opamp){};
            \draw (opamp.+) node[ground]{};
            \draw (2, 0) -- (2, 1);
            \draw (2, 1) to[resistor, l=$R_f$] (4.5, 1);
            \draw (4.5, 1) -- (4.5, -0.5);
            \draw (opamp.out) to [short] ++ (0.5, 0) node[right]{$V_{out}$};
            \label{fig:inverting_amp}
        \end{circuitikz}
        \caption{Circuit B}
    \end{center}
\end{figure}

\begin{figure}[h!]
    \begin{center}
        \begin{circuitikz}[american, transform shape]
            \draw (0,0) node[left]{$V_{in}$} to[short] ++ (1,0)
                node[op amp, noinv input up, anchor=+](opamp){}
                (opamp.-) -- ++(0,-1) coordinate(FB)
                to[R=$R_i$] ++(0,-2) node[ground]{}
                (FB) to[R=$R_f$] (FB -| opamp.out) -- (opamp.out)
                to [short] ++(1,0) node[right]{$V_{out}$};
            % Drawing very similar to "CircuiTikZ 1.7.0 Manual Page 23"
            \label{fig:non_inverting_amp}
        \end{circuitikz}
        \caption{Circuit C}
    \end{center}
\end{figure}

\begin{figure}[h!]
    \begin{center}
        \begin{circuitikz}[american]
            \draw (0, 0) node[left]{$V_{in+}$};
            \draw (0, 0) to[resistor, l=$R_1$, i=$I_{1}$] (2, 0);
            \draw (2, 0) node[op amp, anchor=-](opamp){};
            \draw (2, 0) -- (2, 1);
            \draw (2, 1) to[resistor, l=$R_3$] (4.5, 1);
            \draw (4.5, 1) -- (4.5, -0.5);
            \draw (opamp.out) to [short] ++ (0.5, 0) node[right]{$V_{out}$};
            \draw (0, -1) node[left]{$V_{in-}$};
            \draw (0, -1) to[resistor, l=$R_2$, i=$I_{2}$] (2, -1);
            \draw (2, -1) to[resistor, l=$R_4$] (2, -3);
            \draw (2, -3) node[ground]{};
            \label{fig:difference_amp}
        \end{circuitikz}
        \caption{Circuit D}
    \end{center}
\end{figure}

\spoilerline

\subsection{Operational Amplifiers}
Ideal opamps (\textit{Operational Amplifier}) are active devices defined by $V_{out} = A \cdot (V_{+} - V_{-})$ and $I_{V_{+}} = I_{V_{-}} = 0$ where $A \approx \inf$. When an opamp is connected in a negative feedback configuration, the inputs, $V_{+}$ and $V_{-}$, are considered "virtually shorted", $V_{+} = V_{-}$. Negative feedback in an opamp circuit means there is a current path from it's output, $V_{out}$, to it's inverting terminal, $V_{-}$.

% consider drawing an opamp with terminals labelled. maybe unnecessary

\subsection{Solutions}
Applying this knowledge of the virtual short assumption we see all configurations given in the question feature negative feedback so we can use in our analysis that $V_{+} = V_{-}$. 

\subsubsection{Circuit A: Unity Gain Amplifier}
From the drawn circuit we see $V_{in} = V_{+}$ and $V_{out} = V_{-}$. For the opamp we can apply $V_{out} = A \cdot (V_{+} - V_{-}) = A \cdot (V_{in} - V_{out}) = A \cdot V_{in} - A \cdot V_{out}$ which can be rearranged into $V_{out} + A \cdot V_{out} = A \cdot V_{in}$, $V_{out} \cdot (1 + A) = A \cdot V_{in}$. Now to solve for the transfer function gives $\frac{V_{out}}{V_{in}} = \frac{A}{1+A}$. Recall that $A \approx \inf$ so the $1$ in the denominator becomes insignificant leading to $\frac{V_{out}}{V_{in}} = 1$ or $V_{out} = V_{in}$.

This circuit can also be solved easily with the virtual short assumption, $V_{+} = V_{-}$, by substituting in $V_{in} = V_{+}$ and $V_{out} = V_{-}$, the solution is trivially $V_{out} = V_{in}$.

This circuit acts as a current buffer. Because $I_{V_{+}} = 0$ and $V_{in} = V_{+}$ we see that $I_{V_{in}} = 0$ so the circuit doesn't load the input at all. However, the op amp is capable of providing however much current is required to any load! 

\subsubsection{Circuit B: Inverting Amplifier}
From the circuit $V_{+} = 0$ and there is a current path $I$ from $V_{in}$ to $V_{out}$. Recall that $I_{in} = 0$ so no current goes into the opamp input pins. Analyzing the resistors gives $V_{in} - V_{-} = I \cdot R_i$ and $V_{-} - V_{out} = I \cdot R_f$. Note that the resistors could be solved with $I$ in the reverse direction as long as equations are internally consistent: $V_{out} - V_{-} = I \cdot R_f$ and $V_{-} - V_{in} = I \cdot R_i$. Next the resistor equations can be substituted into each other to cancel out $I$ so $\frac{V_{out} - V_{-}}{R_f} = \frac{V_{-} - V_{in}}{R_i}$. Simplifying this gives $V_{out} \cdot R_i - V_{-} \cdot R_i = V_{-} \cdot R_f - V_{in} \cdot R_f$. 

Next $V_{out} = A \cdot (V_{+} - V_{-})$ can be applied where $V_{+} = 0$ from the circuit so $V_{out} = -A \cdot V_{-}$ and $V_{-} = -\frac{V_{out}}{A}$. This can be substituted into the resistor equation $V_{out} \cdot R_i - V_{-} \cdot R_i = V_{-} \cdot R_f - V_{in} \cdot R_f$ to get $V_{out} \cdot R_i - (-\frac{V_{out}}{A}) \cdot R_i = (-\frac{V_{out}}{A}) \cdot R_f - V_{in} \cdot R_f$. Next this can be rearranged to find the transfer function: $V_{out} \cdot R_i + \frac{V_{out} \cdot R_i}{A} = -\frac{V_{out} \cdot R_f}{A} - V_{in} \cdot R_f$, $V_{out} \cdot R_i \cdot A + V_{out} \cdot R_i = -V_{out} \cdot R_f - V_{in} \cdot R_f \cdot A$, $V_{out} \cdot R_i \cdot A + V_{out} \cdot R_i + V_{out} \cdot R_f = - V_{in} \cdot R_f \cdot A$, $V_{out} \cdot (R_i \cdot A + R_i + R_f) = - V_{in} \cdot R_f \cdot A$, and $\frac{V_{out}}{V_{in}} = \frac{-R_f \cdot A}{R_i \cdot A + R_i + R_f} = -\frac{R_f}{R_i + \frac{R_i}{A} + \frac{R_f}{A}}$. This can be simplified assuming $A \approx \inf$ so $\frac{1}{A} \approx 0$ resulting in $\frac{V_{out}}{V_{in}} = -\frac{R_f}{R_i}$. 

Alternatively using the virtual short assumption, $V_{+} = V_{-}$, after seeing negative feedback in the circuit. From the circuit $V_{+} = 0$ so $V_{-} = V_{+} = 0$. Plugging this into the resistor equation gives $V_{out} \cdot R_i - 0 \cdot R_i = 0 \cdot R_f - V_{in} \cdot R_f$, $V_{out} \cdot R_i = - V_{in} \cdot R_f$, and $\frac{V_{out}}{V_{in}} = -\frac{R_f}{R_i}$. This approach is much simpler than solving the entire circuit with $A$ and will be used to solve the subsequent circuits as they also exhibit negative feedback.  

\subsubsection{Circuit C: Non-inverting Amplifier}
From the circuit $V_{in} = V_{+}$. From the resistors, $V_{out} - V_{-} = I \cdot R_f$ and $V_{-} - 0 = I \cdot R_i$. Using substitution to cancel out $I$ gives $\frac{V_{out} - V_{-}}{R_f} = \frac{V_{-}}{R_i}$ and $V_{out} \cdot R_i - V_{-} \cdot R_i = V_{-} \cdot R_f$.

While this could be solved using $V_{out} = A \cdot (V_{+} - V_{-})$, this solution will jump to applying the virtual short assumption, $V_{+} = V_{-}$, because negative feedback is present in this configuration. This assumption gives $V_{-} = V_{+} = V_{in}$ which gives $V_{out} \cdot R_i - V_{in} \cdot R_i = V_{in} \cdot R_f$, $V_{out} \cdot R_i = V_{in} \cdot R_f + V_{in} \cdot R_i$, and $\frac{V_{out}}{V_{in}} = \frac{R_f + R_i}{R_i} = frac{R_f}{R_i} + \frac{R_i}{R_i} = 1 + \frac{R_f}{R_i}$. 

Notice when $R_i = \inf \Omega$ and $R_f = 0 \Omega$ the non-inverting amplifier circuit becomes a unity gain amplifier circuit and the transfer function mathematically aligns with this conclusion as $\frac{R_f}{R_i} = \frac{0}{\inf} = 0$ so $\frac{V_{out}}{V_{in}} = 1$. 

\subsubsection{Circuit D: Difference Amplifier}
% TODO write
% solve only with virtual short assumption. 

% consider speacil case of resistors that simplifies things

\subsection{Non-idealities}
While most interview questions focus on ideal opamps, real opamps suffer from numerous non-idealities. The most important of which should be understood to aid in circuit design questions and are described below. 

\subsubsection{Saturation}
Saturation occurs when 
For a rail to rail opamp, $V_{out}$ is bounded by the rails, $V_{ee}$ and $V_{cc}$, such that $V_{ee} < V_{out} < V_{cc}$. 
% TODO write explanation

\subsubsection{Input Offset Voltage}
The input offset voltage is the minimum voltage difference that an amplifier can detect between $V_{+}$ and $V_{-}$ input pins. 

\subsubsection{Input Bias Current}
Input bias current is current, $I_{in}$, into $V_{+}$ and $V_{-}$. This current is usually very small, but has an affect on the circuit! 

\subsection{Follow-ups}
\begin{itemize}
    \item Why are non-inverting amplifiers preferred over inverting amplifiers? % Input Impedance
    \item What is an Instrumentation Amplifier? When would you use it? % Non-inverting amplifier's buffering a difference amplifier
\end{itemize}

\end{document}
