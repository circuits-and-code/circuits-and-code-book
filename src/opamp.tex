\documentclass[main.tex]{subfiles}

\begin{document}

\section{Solve the transfer function for the following circuit.} \label{section:opamp}

% \noindent Note: in a real interview scenario, it's reasonable to only expect just one of these circuits.
% Numerous circuits were initially proposed for this question for the sake of building intuition and the showing first principles approach, however, it is optimal to show a single circuit to reduce scope creeping this question

\begin{figure}[H]
    \begin{center}
        \begin{circuitikz}[american, transform shape]
            \draw (0,0) node[left]{$V_{in}$} to[short] ++ (1,0)
                node[op amp, noinv input up, anchor=+](op-amp){}
                (op-amp.-) -- ++(0,-1) coordinate(FB)
                to[R=$R_i$] ++(0,-2) node[ground]{}
                (FB) to[R=$R_f$] (FB -| op-amp.out) -- (op-amp.out)
                to [short] ++(1,0) node[right]{$V_{out}$};
        \end{circuitikz}
        \caption{Given Circuit}
        \label{fig:non_inverting_amp}
    \end{center}
\end{figure}
% Chose this circuit because it can be simplified into unity gain amplifier and isn't as complicated as the difference amplifier

\spoilerline

\subsection{Operational Amplifiers}
Ideal operational amplifiers (\textit{op-amps}) are active devices characterized by the following equations:

\begin{equation}
    \begin{aligned}
        V_{out} &= A \cdot (V_{+} - V_{-}) \quad \text{where} \quad A \to \infty, \\
        I_{+} &= I_{-} = 0.
    \end{aligned}
    \label{eq:op-amp-governing-equations}
\end{equation}

\begin{figure}[H]
    \begin{center}
        \begin{circuitikz}[american, transform shape]
            \draw (0,0) node[op amp, yscale=-1] (opamp) {};
            \draw (opamp.out) -- ++ (0.5, 0) node[right]{$V_{out}$};
            \draw (opamp.out) to[short, i=$I_{out}$] ++ (0.5, 0);
            \draw (opamp.+) -- ++ (-0.5, 0) node[left]{$V_{+}$} to[short, i=$I_{+}$] (opamp.+);
            \draw (opamp.-) -- ++ (-0.5, 0) node[left]{$V_{-}$} to[short, i=$I_{-}$] (opamp.-);
        \end{circuitikz}
        \caption{Labelled Op-amp}
        \label{fig:labelled_op-amp}
    \end{center}
\end{figure}

\subsubsection{Virtual Short}
\noindent When an op-amp is connected in a negative feedback configuration, the inputs ($V_{+}$ and $V_{-}$) are considered "virtually shorted", meaning $V_{+} = V_{-}$. Negative feedback in an op-amp circuit means there is a current path from it's output, $V_{out}$, to it's inverting terminal, $V_{-}$.

\subsection{Solving}
\noindent When analyzing circuits, begin by defining equations for simple elements and build from there. In this circuit,  Ohms' law can be applied to the $R_f$ and $R_i$ resistors, which results in the following relationships: $V_{out} - V_{-} = I \cdot R_f$ and $V_{-} = I \cdot R_i$. Note in this case, $I_{-} = 0$ as the current ($I$) through resistor $R_f$ is the same as the current through $R_i$. \newline

\newnoindentpara Next, because $R_f$ connects a current path from $V_{out}$ to $V_{-}$, the virtual short assumption holds for the op-amp - therefore, $V_{+} = V_{-}$. From the circuit, it's seen that $V_{in} = V_{+}$, so $V_{in} = V_{+} = V_{-}$. This conclusion can be substituted into the above-derived resistor equations to get $V_{out} - V_{in} = I \cdot R_f$ and $V_{in} = I \cdot R_i$. \newline

\newnoindentpara These equations can be substituted into each other to cancel out $I$ and algebraically rearranged to determine $\frac{V_{out}}{V_{in}} = 1 + \frac{R_f}{R_i}$. This solution for $\frac{V_{out}}{V_{in}}$ is known as the transfer function of the circuit. The concept of a transfer function is used to analyze numerous circuits. 

\subsubsection{Intuition}
To understand this circuit better, consider how this circuit operates.
\begin{itemize}
    \item Because negative resistors don't exist, $R_f > 0$ and $R_i > 0$, $\frac{V_{out}}{V_{in}} > 1$ always. As the gain of the circuit is always positive, this circuit is referred to as a \textit{Non-inverting Amplifier}. 
    \item When $R_f = R_i = R$ the transfer function simplifies into $\frac{V_{out}}{V_{in}} = 2$.
    \item When $R_f >> R_i$ the gain becomes very large, $\frac{V_{out}}{V_{in}} \approx \infty$.
    \item When $R_i >> R_f$ the gain approaches unity, $\frac{V_{out}}{V_{in}} \approx 1$.
\end{itemize}

\subsection{Unity Gain Amplifier}
A special case of this circuit occurs when $R_f$ is replaced with a short circuit, $R_f = 0$, and $R_i$ is replaced with an open circuit, $R_i = \infty$ as drawn in Figure \ref{fig:unity-amp}.

\begin{figure}[H]
    \begin{center}
        \begin{circuitikz}[american, transform shape]
        \draw (0,0) node[left]{$V_{in}$} to[short] ++ (1,0)
            node[op amp, noinv input up, anchor=+](op-amp){}
            (op-amp.-) -- ++(0,-1) coordinate(FB)
            (FB) to[short] (FB -| op-amp.out) -- (op-amp.out)
            to [short] ++(1,0) node[right]{$V_{out}$};
        \end{circuitikz}
        \caption{Unity Gain Amplifier Circuit}
        \label{fig:unity-amp}
    \end{center}
\end{figure}

\noindent In this case the transfer function becomes unity, $\frac{V_{out}}{V_{in}} = 1$ which can be simplified into $V_{out} = V_{in}$. \newline

\newnoindentpara This circuit acts as a current buffer. Because $I_{V_{+}} = 0$ and $V_{in} = V_{+}$ we see that $I_{in} = 0$ so the circuit doesn't load the input at all. Instead, any current required by the load is provided by the op-amp! 

% \subsection{Non-idealities}
% While most interview questions focus on ideal op-amps, real op-amps suffer from numerous non-idealities. The most important of which should be understood to aid in circuit design questions and are described below. 

% \subsubsection{Saturation}
% Saturation occurs when 
% For a rail to rail op-amp, $V_{out}$ is bounded by the rails, $V_{ee}$ and $V_{cc}$, such that $V_{ee} < V_{out} < V_{cc}$. % more details required here to explain

% \subsubsection{Input Offset Voltage}
% The input offset voltage is the minimum voltage difference that an amplifier can detect between $V_{+}$ and $V_{-}$ input pins. 

% \subsubsection{Input Bias Current}
% Input bias current is current, $I_{in}$, into $V_{+}$ and $V_{-}$. This current is usually very small, but has an affect on the circuit! 

\subsection{Follow-ups}
\begin{itemize}
    \item What is an inverting amplifier?
    \item What is a difference amplifier?
    \item Describe the non-idealities of op-amps and how to compensate for them? % Saturation, Input offset voltage, input bias current 
    \item Why are non-inverting amplifiers preferred over inverting amplifiers? % Input Impedance
    \item Solve the circuits in appendix Section \ref{appendix:more_opamps}. 
    % \item What is an Instrumentation Amplifier? When would you use it? % Non-inverting amplifier's buffering a difference amplifier
\end{itemize}

\end{document}
