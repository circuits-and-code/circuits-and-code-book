\documentclass[main.tex]{subfiles}
\begin{document}

\section{Appendix: Solve the transfer function of the following circuits.} \label{appendix:more_opamps}

\noindent The following circuits are similar to those in Section \ref{section:opamp} and are given as extra excersices with final answers only for checking.

\begin{figure}[H]
    \begin{center}
        \begin{minipage}{0.45\textwidth}
            \centering
            \begin{circuitikz}[american]
                \draw (0,0) node[left]{$V_{in}$} to[short] ++ (1,0)
                    node[op amp, noinv input up, anchor=+](op-amp){}
                    (op-amp.-) -- ++(0,-1) coordinate(FB)
                    (FB) to[short] (FB -| op-amp.out) -- (op-amp.out)
                    to [short] ++(1,0) node[right]{$V_{out}$};
            \end{circuitikz}
            \caption{Circuit A}
            \label{fig:unity-amp}
        \end{minipage}%
        \hfill%
        \begin{minipage}{0.45\textwidth}
            \centering
            \begin{circuitikz}[american]
                \draw (0, 0) node[left]{$V_{in}$};
                \draw (0, 0) to[resistor, l=$R_i$] (2, 0);
                \draw (2, 0) node[op amp, anchor=-](op-amp){};
                \draw (op-amp.+) node[ground]{};
                \draw (2, 0) -- (2, 1);
                \draw (2, 1) to[resistor, l=$R_f$] (4.5, 1);
                \draw (4.5, 1) -- (4.5, -0.5);
                \draw (op-amp.out) to [short] ++ (0.5, 0) node[right]{$V_{out}$};
            \end{circuitikz}
            \caption{Circuit B}
            \label{fig:inverting_amp}
        \end{minipage}
    \end{center}
\end{figure}
\begin{figure}[H]
    \begin{center}
        \begin{minipage}{0.45\textwidth}
            \centering
            \begin{circuitikz}[american, transform shape]
                \draw (0,0) node[left]{$V_{in}$} to[short] ++ (1,0)
                    node[op amp, noinv input up, anchor=+](op-amp){}
                    (op-amp.-) -- ++(0,-1) coordinate(FB)
                    to[R=$R_i$] ++(0,-2) node[ground]{}
                    (FB) to[R=$R_f$] (FB -| op-amp.out) -- (op-amp.out)
                    to [short] ++(1,0) node[right]{$V_{out}$};
            \end{circuitikz}
            \caption{Circuit C}
            \label{fig:non_inverting_amp}
        \end{minipage}%
        \hfill%
        \begin{minipage}{0.45\textwidth}
            \centering
            \begin{circuitikz}[american]
                \draw (0, 0) node[left]{$V_{in-}$};
                \draw (0, 0) to[resistor, l=$R_1$, i=$I_{1}$] (2, 0);
                \draw (2, 0) node[op amp, anchor=-](op-amp){};
                \draw (2, 0) -- (2, 1);
                \draw (2, 1) to[resistor, l=$R_3$] (4.5, 1);
                \draw (4.5, 1) -- (4.5, -0.5);
                \draw (op-amp.out) to [short] ++ (0.5, 0) node[right]{$V_{out}$};
                \draw (0, -1) node[left]{$V_{in+}$};
                \draw (0, -1) to[resistor, l=$R_2$, i=$I_{2}$] (2, -1);
                \draw (2, -1) to[resistor, l=$R_4$] (2, -3);
                \draw (2, -3) node[ground]{};
            \end{circuitikz}
            \caption{Circuit D}
            \label{fig:difference_amp}
        \end{minipage}
    \end{center}
\end{figure}

\spoilerline

\begin{itemize}
    \item Circuit A is a unity gain amplifier where $\frac{V_{out}}{V_{in}} = 1$.
    \item Circuit B is an inverting amplifier where $\frac{V_{out}}{V_{in}} = -\frac{R_f}{R_i}$.
    \item Circuit C is a non-inverting amplifier where $\frac{V_{out}}{V_{in}} = 1 + \frac{R_f}{R_i}$.
    \item Circuit D is a differential amplifier where $V_{out} = -V_{in-} \cdot \frac{R_3}{R_1} + V_{in+} \cdot \frac{R_4}{R_2 + R_4} \cdot \frac{R_1 + R_3}{R_1}$. Consider a speacil case when $R_1 = R_2$ and $R_3 = R_4$ where the transfe function becomes $\frac{V_{out}}{V_{in+} - V_{in-}} = \frac{R_3}{R_1}$. Additionally, if $R_1 = R_2 = R_3 = R_4$ then $V_{out} = V_{in+} - V_{in-}$.
\end{itemize}

% the text below that i interleaved with math is quite hard to read, consider a better approach to demonstrate these equarions

% \subsection{Solutions}
% Given all of the circuit configurations feature negative feedback, the virtual short assumption can be applied to simplify the analysis (that is, $V_{+} = V_{-}$).

% \subsubsection{Circuit A: Unity Gain Amplifier}
% From the drawn circuit we see $V_{in} = V_{+}$ and $V_{out} = V_{-}$. For the op-amp we can apply $V_{out} = A \cdot (V_{+} - V_{-}) = A \cdot (V_{in} - V_{out}) = A \cdot V_{in} - A \cdot V_{out}$ which can be rearranged into $V_{out} + A \cdot V_{out} = A \cdot V_{in}$, $V_{out} \cdot (1 + A) = A \cdot V_{in}$. Now to solve for the transfer function gives $\frac{V_{out}}{V_{in}} = \frac{A}{1+A}$. Recall that $A \approx \inf$ so the $1$ in the denominator becomes insignificant leading to $\frac{V_{out}}{V_{in}} = 1$ or $V_{out} = V_{in}$.

% This circuit can also be solved easily with the virtual short assumption, $V_{+} = V_{-}$, by substituting in $V_{in} = V_{+}$ and $V_{out} = V_{-}$, the solution is trivially $V_{out} = V_{in}$.

% This circuit acts as a current buffer. Because $I_{V_{+}} = 0$ and $V_{in} = V_{+}$ we see that $I_{V_{in}} = 0$ so the circuit doesn't load the input at all. However, the op amp is capable of providing however much current is required to any load! 

% \subsubsection{Circuit B: Inverting Amplifier}
% From the circuit $V_{+} = 0$ and there is a current path $I$ from $V_{in}$ to $V_{out}$. Recall that $I_{in} = 0$ so no current goes into the op-amp input pins. Analyzing the resistors gives $V_{in} - V_{-} = I \cdot R_i$ and $V_{-} - V_{out} = I \cdot R_f$. Note that the resistors could be solved with $I$ in the reverse direction as long as equations are internally consistent: $V_{out} - V_{-} = I \cdot R_f$ and $V_{-} - V_{in} = I \cdot R_i$. Next the resistor equations can be substituted into each other to cancel out $I$ so $\frac{V_{out} - V_{-}}{R_f} = \frac{V_{-} - V_{in}}{R_i}$. Simplifying this gives $V_{out} \cdot R_i - V_{-} \cdot R_i = V_{-} \cdot R_f - V_{in} \cdot R_f$. 

% Next $V_{out} = A \cdot (V_{+} - V_{-})$ can be applied where $V_{+} = 0$ from the circuit so $V_{out} = -A \cdot V_{-}$ and $V_{-} = -\frac{V_{out}}{A}$. This can be substituted into the resistor equation $V_{out} \cdot R_i - V_{-} \cdot R_i = V_{-} \cdot R_f - V_{in} \cdot R_f$ to get $V_{out} \cdot R_i - (-\frac{V_{out}}{A}) \cdot R_i = (-\frac{V_{out}}{A}) \cdot R_f - V_{in} \cdot R_f$. Next this can be rearranged to find the transfer function: $V_{out} \cdot R_i + \frac{V_{out} \cdot R_i}{A} = -\frac{V_{out} \cdot R_f}{A} - V_{in} \cdot R_f$, $V_{out} \cdot R_i \cdot A + V_{out} \cdot R_i = -V_{out} \cdot R_f - V_{in} \cdot R_f \cdot A$, $V_{out} \cdot R_i \cdot A + V_{out} \cdot R_i + V_{out} \cdot R_f = - V_{in} \cdot R_f \cdot A$, $V_{out} \cdot (R_i \cdot A + R_i + R_f) = - V_{in} \cdot R_f \cdot A$, and $\frac{V_{out}}{V_{in}} = \frac{-R_f \cdot A}{R_i \cdot A + R_i + R_f} = -\frac{R_f}{R_i + \frac{R_i}{A} + \frac{R_f}{A}}$. This can be simplified assuming $A \approx \inf$ so $\frac{1}{A} \approx 0$ resulting in $\frac{V_{out}}{V_{in}} = -\frac{R_f}{R_i}$. 

% Alternatively using the virtual short assumption, $V_{+} = V_{-}$, after seeing negative feedback in the circuit. From the circuit $V_{+} = 0$ so $V_{-} = V_{+} = 0$. Plugging this into the resistor equation gives $V_{out} \cdot R_i - 0 \cdot R_i = 0 \cdot R_f - V_{in} \cdot R_f$, $V_{out} \cdot R_i = - V_{in} \cdot R_f$, and $\frac{V_{out}}{V_{in}} = -\frac{R_f}{R_i}$. This approach is much simpler than solving the entire circuit with $A$ and will be used to solve the subsequent circuits as they also exhibit negative feedback.  

% \subsubsection{Circuit C: Non-inverting Amplifier}
% From the circuit $V_{in} = V_{+}$. From the resistors, $V_{out} - V_{-} = I \cdot R_f$ and $V_{-} - 0 = I \cdot R_i$. Using substitution to cancel out $I$ gives $\frac{V_{out} - V_{-}}{R_f} = \frac{V_{-}}{R_i}$ and $V_{out} \cdot R_i - V_{-} \cdot R_i = V_{-} \cdot R_f$.

% While this could be solved using $V_{out} = A \cdot (V_{+} - V_{-})$, this solution will jump to applying the virtual short assumption, $V_{+} = V_{-}$, because negative feedback is present in this configuration. This assumption gives $V_{-} = V_{+} = V_{in}$ which gives $V_{out} \cdot R_i - V_{in} \cdot R_i = V_{in} \cdot R_f$, $V_{out} \cdot R_i = V_{in} \cdot R_f + V_{in} \cdot R_i$, and $\frac{V_{out}}{V_{in}} = \frac{R_f + R_i}{R_i} = frac{R_f}{R_i} + \frac{R_i}{R_i} = 1 + \frac{R_f}{R_i}$. 

% Notice when $R_i = \inf \Omega$ and $R_f = 0 \Omega$ the non-inverting amplifier circuit becomes a unity gain amplifier circuit and the transfer function mathematically aligns with this conclusion as $\frac{R_f}{R_i} = \frac{0}{\inf} = 0$ so $\frac{V_{out}}{V_{in}} = 1$. 

% \subsubsection{Circuit D: Difference Amplifier}
% Each resistor can be analyzed using Ohm's law: 
% \begin{itemize}
%     \item $V_{in+} - V_{-} = I_1 \cdot R_1$
%     \item $V_{-} - V_{out} = I_1 \cdot R_3$
%     \item $V_{in-} - V_{+} = I_2 \cdot R_2$
%     \item $V_{+} - 0 = I_2 \cdot R_4$
% \end{itemize}
% Substituting to cancel $I_1$ and $I_2$ gives:
% \begin{itemize}
%     \item $\frac{V_{in+} - V_{-}}{R_1} = \frac{V_{-} - V_{out}}{R_3}$
%     \item $\frac{V_{in-} - V_{+}}{R_2} = \frac{V_{+}}{R_4}$
% \end{itemize}
% As negative feedback is depicted via $R_3$ connecting $V_{out}$ and $V_{-}$ the virtual short assumption holds for this circuit so $V_{+} = V_{-}$. 
% \begin{itemize}
%     \item $(V_{in+} - V_{+}) \cdot R_3 = (V_{+} - V_{out}) \cdot R_1$
%     \item $(V_{in-} - V_{+}) \cdot R_4 = V_{+} \cdot R_2$
% \end{itemize}
% Now these can be substituted into each other to cancel out $V_{+}$. 

\end{document}
