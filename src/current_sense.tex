\documentclass[main.tex]{subfiles}
\begin{document}

\section{How would you sense how much current is flowing through a PCB to a load?}

\subsection{Motivation}
Embedded systems often need to track power and current consumption of loads to:
\begin{itemize}
    \item Determine battery state of charge by integrating the power the battery has been charged and discharged over time
    \item Detect if a load is drawing an excessive amount of power indicating it has failed or an anomaly is occurring. 
    \item Ensure the system does not overload a source power supply which could cause the entire system to lose power. 
    \item Ensure a load is connected nominally.
\end{itemize}
Power and current are electrically related by $P = I * V$. We've already seen how sense microcontroller's can sense voltages, $V$, so now we will explore current, $I$, sensing which is related to power electrically where $P = I * V$.

\subsection{Resistive Current Sensing}
The simplest and most commonly used method of current sense in embedded systems is resistive current sense which leverages Ohm's Law, $I = \frac{V}{R}$, to sense the voltage drop across a known resistor value in series with the load to determine the amount of current flowing.
An important design decision for engineers is selecting the resistor to use as $R$ in this case. 

% If the value chosen for $R$ is large -power consumption goes up
% large vs small R values 

% high side vs low side 



\subsection{Other Methods}
% TODO : write

\end{document}
