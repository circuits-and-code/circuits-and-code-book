\documentclass[main.tex]{subfiles}
\begin{document}

\section{If you probe the resistance a 50 ohm coaxial cable on one end with the other end open circuit with a DMM, what resistance does it show? What if the cable is infinitely long?}

\spoilerline

% image: what probing the coaxial cable looks like (center pin and shield), real or hand drawn is ight i guess?

\noindent On a bench-top coaxial cable (relatively short length) the DMM (\textit{Digital Multi Meter}) will read OL (\textit{Open Loop}) when probed as in DC analysis the center pin is not connected to the outer shield. When the transmission line is of infinite length the DMM will report the characteristic impedance, in this case, 50 ohms.

\subsection{Transmission Lines}
Here we will analyze the lossless transmission line model in which conductors are considered without resistance and dielectrics are considered without loss. This model is drawn to represent a transmission line in which the conductor has inductance and the two conductors have capacitance to each other, each represented per unit length as $L\prime$ and $C\prime$ respectively.
% draw : schematic lumped element section of LC inside transmission line model (ece 373 lec 1 slide 17 has this proof)
Analyzing one small section of this line for $\Delta z$ 
% TODO : basically use lumped element model to derive/prove Z_o and beta definitions. 
$Z_o = \sqrt{\frac{L}{C}}$ and $\beta = \omega * \sqrt{L*C}$.

\subsection{DMM Resistance Measurement}
A DMM measured resistance by 
% TOOD : write 

In the case of the finite length transmission line, the DMM and human are usually not fast enough to sample the resistance before the transmission line is able to refelct the voltage waves back and forth down the transmission line enough for it to all even out and only the DC values to be considerred.

In the case of the infinite length transmission line, the DMM is creating a voltage wave down the transmission line that is never reflected back (as the line is infinite) and is measuring the current it is providing into the line. 

\subsection{Time Domain Reflectometry}
Lossless transmission lines are characerized by a propogation constant, $beta$, and characteristic impedance, $Z_o$. 

A finite length coaxial cable can have it's length measured using time domain reflectometry (TDR) analysis provided the propogation constant is known. This is useful as it let's us figure out where a break could be inside a coaxial cable. 

% TODO : consider, concept of a bounce diagram to be discussed here? ? 

\subsection{Types of Transmission Lines}
Transmission line types:
\begin{itemize}
    \item Coaxial cable
    \item PCB trace
    \item Waveguide
\end{itemize}
Most common forms of PCB Transmission Lines are:
\begin{itemize}
    \item Microstrip
    \item Stripline
    \item Coplanar
\end{itemize}
Adjusting the characteristic impedance of a microstrip transmission line can be done by:
\begin{itemize}
    \item Decreasing trace width results in less capacitance per unit length 
    \item Decreasing layer spacing results in increased capacitance per unit length and therefore decreasing $Z_o$.
\end{itemize}


\end{document}
