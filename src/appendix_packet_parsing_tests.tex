\documentclass[main.tex]{subfiles}
\begin{document}

\section{Extra Practice: Write a unit test for a packet parsing function.} \label{extra_practice:packet_parsing_tests}

\spoilerline

\subsection{Solution}
\noindent As mentioned in Question \ref{section:packet_parsing} (\textit{\nameref{section:packet_parsing}}), unit tests are a crucial part of software development. They help ensure that the code behaves as expected and catches bugs early in the development process. In this section, an example of a unit test framework and unit tests for the packet parsing function are presented. Note that the tests and framework are crude to demonstrate the type of code that would be written in an interview setting. In real-life, a dedicated test framework, like CppUTest or GoogleTest, is strongly recommended. \newline

\newnoindentpara The unit tests in principle work by the expected output being the extracted values as well as the return value of the parsing function. Listing \ref{code:parsing_test_function} shows example unit tests and crude test framework for testing the packet parsing function.

\lstinputlisting[caption={Parsing Test Function}, label={code:parsing_test_function}]{code/packet_parsing/main.c}

\end{document}
