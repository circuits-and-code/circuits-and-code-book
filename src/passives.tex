\documentclass[main.tex]{subfiles}
\begin{document}

\section{Determine the step response of the following circuits.}

\noindent In a real interview you would likely be given just one of these circuits.
% Numerous circuits are analyzed for the sake of building intuition and the showing first principles approach.

\begin{figure}[h!]
    \begin{center}
        \begin{circuitikz}[american]
            \draw (0, 4) node[anchor=east] {$V_{in}$};
            \draw (0, 4) to[resistor, l=$R$] (5, 4);
            \draw (5, 4) node[right] {$V_{out}$};
            \draw (-0.5, 2) -- (5.5, 2); % ground line at the bottom
            \draw (3, 2) node[ground]{};
            \label{ct:series_resistor}
        \end{circuitikz}
        \caption{Circuit A}
    \end{center}
\end{figure}

\begin{figure}[h!]
    \begin{center}
        \begin{circuitikz}[american]
            \draw (0, 4) node[anchor=east] {$V_{in}$};
            \draw (0, 4) to[resistor, l=$R_1$] (5, 4); % top resistor to the left, horizontal
            \draw (3.5, 4) to[resistor, l=$R_2$] (3.5, 2); % bottom resistor, vertical
            % \draw (4, 4) -- (5, 4); % line from the resistor to V_{out} net, not needed how i drew it! 
            \draw (5, 4) node[right] {$V_{out}$};
            \draw (-0.5, 2) -- (5.5, 2); % ground line at the bottom
            \draw (3, 2) node[ground]{};
            \label{fig:voltage_divider}
        \end{circuitikz}
        \caption{Circuit B}
    \end{center}
\end{figure}

\begin{figure}[h!]
    \begin{center}
        \begin{circuitikz}[american]
            % TODO DRAW
            \label{fig:rc_low_pass_filter}
        \end{circuitikz}
        \caption{Circuit C}
    \end{center}
\end{figure}

\begin{figure}[h!]
    \begin{center}
        \begin{circuitikz}[american]
            % TODO DRAW
            \label{fig:rc_high_pass_filter}
        \end{circuitikz}
        \caption{Circuit D}
    \end{center}
\end{figure}

\begin{figure}[h!]
    % TODO DRAW
    \caption{Circuit E}
    \label{fig:c_current_source}
\end{figure}

\begin{figure}[h!]
    % TODO DRAW
    \caption{Circuit F}
    \label{fig:rc_current_source}
\end{figure}

\begin{figure}[h!]
    % TODO DRAW
    \caption{Circuit G}
    \label{fig:lseries_cshunt} 
\end{figure}

\begin{figure}[h!]
    % TODO DRAW
    \caption{Circuit H}
    \label{fig:lseries_cseries} 
\end{figure}

\spoilerline

\noindent For a question like this, assuming ideal components is expected as further information has not been provided to aid in a more complex analysis.

\subsection{Passives}
Recall all ideal passives follow the definitions given in the following figure. Note that $\omega = 2 * \pi * f$ and the imaginary component of impedance, $Z = \frac{V}{I}$ represents reactance $X$ defining a phase shift. Resistors are not frequency or time dependent while inductors and capacitors are. 
\begin{figure}[h!]
    \centering
    \begin{tabular}{|c|c|c|c|}
        \hline
         & Resistor & Capacitor & Inductor \\ \hline
        Time Domain & $V = I * R$ & $I = C * \frac{dv}{dt}$ & $V = L * \frac{di}{dt}$ \\ \hline
        Frequency Domain & $\frac{V}{I} = R$ & $\frac{V}{I} = \frac{1}{j*\omega*C}$ & $\frac{V}{I} = j*\omega*L$ \\ \hline
    \end{tabular}
    \caption{Passive Element Definitions}
    \label{fig:passive definitions}
\end{figure}

\subsection{Circuit A: Series Resistor}
In this circuit we note Ohm's law, $V = I * R$, however, as there is no load, $I=0$ on the resistor in this circuit we note there is no voltage drop, $V=I*R=0*R=0$, across the resistor, consequently $V_{in} = V_{out}$ for this circuit. 

% DRAW: a plot just showing an input step being followed perfectly

\subsection{Circuit A: Voltage Divider}
This circuit is a voltage divider depicted slightly differently than in a previous question, but the same equations hold. 


\subsection{Circuit C: RC Low Pass Filter}
From the two elements in the circuit we get in the time domain $I = C * \frac{dV_{out}}{dt}$ and $V_{in} - V_{out} = I * R$. Cancelling out $I$ which is equivalent in both elements we get $V_{in} - V_{out} = R * C * \frac{dV_{out}}{dt}$ which can be algebraically manipulated into $-\frac{dV_{out}}{V_{out}-V_{in}} = -\frac{dt}{R*C}$ where we perform the integral on each side revealing $\ln{\frac{V_{out}-V_{in}}{V_out}}=-\frac{t}{R*C}$ which can be algebraically manipulated into $V_{out}=V_{in}*(1-\exp(\frac{-t}{R*C}))$. Note when analyzing this circuit we commonly define a constant $\tau = R * C$ as the time constant of the RC circuit which represents the time it takes $1-\exp{-1}$ or ~63 percent of the step size. 

% TODO : add a nice solution plot showing an exponential rise and some key points labelled, notably time constant 

This circuit can also be analyzed in the frequency domain. 
% reasonable to do a full derivation here without laplace

% TODO : frequency domain plot of this circuit in log log scale 

% connection to I2C

\subsection{Circuit D: RC High Pass Filter}

\subsection{Circuit E: Series Capacitor with Current Source}

\subsection{Circuit F: Series RC with Current Source}

\subsection{Circuit G: Series Inductor with Shunt Capacitor}

\subsection{Circuit H: Series Inductor and Capacitor}

\end{document}
