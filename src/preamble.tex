\documentclass[main.tex]{subfiles}
\begin{document}

\section{Introduction}

This guide was created by two Waterloo Engineering students, \bluehref{https://www.linkedin.com/in/daniel-puratich}{Daniel Puratich} and \bluehref{https://www.linkedin.com/in/sahil-kale}{Sahil Kale}. Wanting to support our peers in their co-op journeys, we noticed that we were often answering the same questions when asked about co-op interview prep. Noticing that many firmware and hardware co-op interviews focus on fundamental concepts, we decided to compile a comprehensive list of commonly asked questions and answers to help students better prepare for their interviews.

\subsection{Target Audience}
Our guide is designed for engineering undergraduates seeking technical internships in embedded software, firmware, and electrical engineering across the US and Canada. The questions and answers aim to provide a foundational understanding, making them ideal for those just starting in the field.

\subsection{How to Use This Guide}
This guide is structured to help you understand and answer common technical questions asked in firmware and hardware co-op interviews. Each section starts by introducing a question and relevant informational content, followed by a detailed answer. We recommend reading through the content and attempting to answer the questions yourself before reviewing the provided answers. Answers begin after this line:
\spoilerline

\noindent Embedded systems cover a broad range of topics that intersect hardware and software. While most questions in this guide are broadly applicable to both firmware and hardware roles, some are tailored to specific industries. For example, topics like mutexes versus semaphores are more likely to appear in firmware interviews, whereas transmission line impedance is more likely to appear in EE interviews. We recommend focusing on the questions most relevant to your target role while gaining a general understanding of both fields.

\subsubsection{Code}
Code snippets are included to illustrate various embedded software concepts. The code is written in C and can be copied and compiled at your discretion, though some snippets may require the inclusion of additional header files and minor restructuring. Note that while snippets can be compiled, they are not intended to be standalone programs or run unless a \texttt{main} function is defined.

\subsection{Acronyms}
This guide features numerous acronyms and phrases that are commonly accepted in industry. We will provide definitions for these acronyms the first time we use them, but will assume you understand them in subsequent questions as understanding these terms will be integral to understanding common interview questions. 

\newpage
\subsection{About the Authors}
\bluehref{https://www.linkedin.com/in/sahil-kale}{Sahil} interned at Tesla, Skydio, and BETA Technologies, focusing on real-time embedded software for safety-critical control systems.
\newline
\newline
\bluehref{https://www.linkedin.com/in/daniel-puratich}{Daniel} interned at Tesla, Anduril Industries, and Pure Watercraft, focusing on power electronics and board design. \newline

\newnoindentpara We met at (and both led) the Waterloo Aerial Robotics Group (WARG), a student team that designs and builds autonomous drones. Uniquely, we have experience in hiring and interviewing multiple co-op students, giving us insight into the interview process from both sides, as well as an understanding of what responses are expected from candidates. We value mentorship, enjoy sharing our knowledge, and take pride in helping others succeed in their co-op journeys.

\subsubsection{Some of our other work}
We've published other (free!) guides to help engineering students land firmware and hardware co-op roles. Check them out:
\begin{itemize}
    \item \bluehref{https://docs.google.com/document/d/1Qh0Jp70ce2ParzsWV4TizEAvGyTrDO5WfkacxrHjTFs}{The Sahil and Daniel Co-op Resume Guide} - focuses on resume writing and tailoring for hardware and firmware roles.
    \item \bluehref{https://docs.google.com/document/d/12qFdJfc2ve2jIFIFHqtIzx0qkN6I5H43kfLo0OQ1X_A}{The Sahil and Daniel Co-op Process Guide} - offers our tips and tricks to landing a co-op role in hardware and firmware.
\end{itemize}


\subsection{Acknowledgements}
We'd like to thank the following individuals for taking the time to review and provide feedback on this guide:
\begin{itemize}
    \item Ronak Patel
\end{itemize}

\subsection{Disclaimer}
This book is designed to be an educational resource, drawing from the authors' experiences and research. While we've done our best to ensure accuracy, readers are encouraged to use their own judgment and explore additional resources as needed. The authors and publisher are not responsible for any errors or omissions. Please note, the content is for informational purposes only and is not intended as professional advice.
\newline
\newnoindentpara \textcopyright \ Sahil Kale, Daniel Puratich.


\end{document}
