\documentclass[main.tex]{subfiles}
\begin{document}

\section{Introduction}

This guide was created by two Waterloo Engineering students, \bluehref{https://www.linkedin.com/in/danielpuratich}{Daniel Puratich} and \bluehref{https://www.linkedin.com/in/sahil-kale}{Sahil Kale}. Wanting to support our peers in their co-op journeys, we noticed that we were often answering the same questions when asked about co-op interview prep. Noticing that many firmware and hardware co-op interviews focus on fundamental concepts, we decided to compile a comprehensive list of commonly asked questions and answers to help students better prepare for their interviews.

\subsection{About the Authors}
\bluehref{https://www.linkedin.com/in/sahil-kale}{Sahil} interned at Tesla, Skydio, and BETA Technologies, focusing on real-time embedded software for safety-critical control systems.
\newline
\newline
\bluehref{https://www.linkedin.com/in/danielpuratich}{Daniel} interned at Tesla, Anduril Industries, and Pure Watercraft, focusing on power electronics and board design.

\subsection{Target Audience}
Our guide is designed for engineering undergraduates seeking technical internships in embedded software, firmware, and electrical engineering across the US and Canada. The questions and answers aim to provide a foundational understanding, making them ideal for those just starting in the field.
\newline
\newline
\noindent We both study at the University of Waterloo, where all engineering students complete six co-op/internship terms as part of their undergraduate program. This provides students with invaluable work experience before graduation and the chance to explore different types of roles within their field.

\subsection{Acronyms}
This guide features numerous acronyms and phrases that are commonly accepted in industry. We will provide definitions for these acronyms the first time we use them, but will assume you understand them in subsequent questions as understanding these terms will be integral to understanding common interview questions. 

\end{document}
