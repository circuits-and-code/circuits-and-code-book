\documentclass[main.tex]{subfiles}
\begin{document}

\section{Introduction}

This guide was created by two Waterloo Engineering students, \bluehref{https://www.linkedin.com/in/danielpuratich}{Daniel Puratich} and \bluehref{https://www.linkedin.com/in/sahil-kale}{Sahil Kale}. Wanting to support our peers in their co-op journeys, we noticed that we were often answering the same questions when asked about co-op interview prep. Noticing that many firmware and hardware co-op interviews focus on fundamental concepts, we decided to compile a comprehensive list of commonly asked questions and answers to help students better prepare for their interviews.

\subsection{About the Authors}
\bluehref{https://www.linkedin.com/in/sahil-kale}{Sahil} interned at Tesla, Skydio, and BETA Technologies, focusing on real-time embedded software for safety-critical control systems.
\newline
\newline
\bluehref{https://www.linkedin.com/in/danielpuratich}{Daniel} interned at Tesla, Anduril Industries, and Pure Watercraft, focusing on power electronics and board design.

\subsection{Target Audience}
Our guide is designed for engineering undergraduates seeking technical internships in embedded software, firmware, and electrical engineering across the US and Canada. The questions and answers aim to provide a foundational understanding, making them ideal for those just starting in the field.

\subsection{How to Use This Guide}
This guide is structured to help you understand and answer common technical questions asked in firmware and hardware co-op interviews. Each section starts by introducing a question and relevant informational content, followed by a detailed answer. We recommend reading through the content and attempting to answer the questions yourself before reviewing the provided answers.
\newline
\newnoindentpara Embedded systems cover a broad range of topics that intersect hardware and software. While most questions in this guide are broadly applicable to both firmware and hardware roles, some are tailored to specific industries. For example, topics like mutexes versus semaphores are more likely to appear in firmware interviews, whereas hardware-oriented questions might focus on circuit design or component selection. We recommend focusing on the questions most relevant to your target role while gaining a general understanding of both fields.

\subsubsection{Code}
Code snippets are included to illustrate various embedded software concepts. The code is written in C and can be copied and compiled/run at your discretion, though some snippets may require the inclusion of additional header files and minor restructuring.

\subsection{Acronyms}
This guide features numerous acronyms and phrases that are commonly accepted in industry. We will provide definitions for these acronyms the first time we use them, but will assume you understand them in subsequent questions as understanding these terms will be integral to understanding common interview questions. 

\subsection{Disclaimer}
This book is intended as a resource for educational purposes only. The content is based on the authors' experiences and research, and while every effort has been made to ensure accuracy, the author and publisher do not guarantee specific results or outcomes. Readers are encouraged to use their judgment and seek additional resources where needed. The authors and publisher are not liable for any errors or omissions. The content of this book is for informational purposes only and does not constitute professional advice.

\end{document}
