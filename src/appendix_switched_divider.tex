\documentclass[main.tex]{subfiles}
\begin{document}

\section{Appendix: Propose a simple circuit to disable current consumption when the voltage divider is not needed.} \label{appendix:switched_divider}

\spoilerline

% This is a really important concept for low power systems but is fairly complex
\subsection{Quiescent Current}
Quiescent current is current drawn when a circuit is inactive or not being used and is proportional to power consumption. Reducing the quiescent current of a voltage divider can be done by increasing $R_{sum}$, however, this only gets you so far. For low power devices, transistor based circuits can be used to disable current flow when sampling the voltage divider is not necessary. An example circuit is given in Figure \ref{fig:voltage_divider_switched}.
\begin{figure}[H]
    \begin{center}
        \begin{circuitikz}[american]
            \draw(-1, 3.5) node[pigfete, bodydiode, rotate=90](q1){};
            \draw (-4, 3.5) -- (q1.S);
            \draw (-4, 3.5) node[anchor=east] {$V_{in}$};
            \draw (q1.G) ++ (0, -1.75) node[nigfete, bodydiode](q2){};
            \draw (-3, 3.5) to[resistor, l=$R_{pullup}$] (-3, 1.5);
            \draw (-3, 1.5) -- (q2.D);
            \draw (q2.S) node[ground]{};
            \draw (q2.D) -- (q1.G);
            \draw (-4, 0.5) node[anchor=east] {$V_{ctrl}$};
            \draw (-4, 0.5) -- (q2.G);
            \draw (-0.5, 3.5) -- (0.5, 3.5) -- (0.5, 3); 
            \draw (0.5, 3) to[resistor, l=$R_t$] (0.5, 1.5) to[resistor, l=$R_b$] (0.5, 0);
            \draw (0.5, 0) node[ground]{};
            \draw (0.5, 1.5) -- (1.25, 1.5) node[right] {$V_{out}$};
        \end{circuitikz}
        \caption{Switched Voltage Divider Circuit}
        \label{fig:voltage_divider_switched}
    \end{center}
\end{figure}
% comment, This circuit is also explained in https://electronics.stackexchange.com/questions/64490/low-current-battery-monitoring/64491#64491 & https://discord.com/channels/776618956638388305/779145203688013845/1199193108336889958 . It's fairly interesting so I figure it's worth including in the guide though it is a bit scope creep. In reality dual FETs in a single package can often be found for cheap so an NPN isn't usually used, thats why i drew it this way.

\end{document}
