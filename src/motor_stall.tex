\documentclass[main.tex]{subfiles}

\begin{document}
\section{Suppose a Brushed DC motor is connected to a load - how would you detect a stalled-rotor condition?}
Discuss what sensor(s) could be used to detect a stalled-rotor condition, and provide a C code snippet that implements the algorithm with the proposed sensor. Assume a header similar to the one in listing \ref{code:motor-stall-header} is provided. 

\lstinputlisting[caption={Motor Stall Detection Sample Header}, label={code:motor-stall-header}]{code/motor_stall/motor_stall_header.h}

\spoilerline

\subsection{Defining a Stalled-Rotor Condition}
A stalled-rotor condiition occurs when the motor's rotor can no longer rotate, typically due to an excessive load or mechanical obstruction. If a motor is left in this condition, it may result in overheating (and damage to the motor windings), increased power consumption, and depending on the system, cause damage to other systems or people. Accordingly, implementing stall detection is a classic requirement for active control systems that use motors.

\subsubsection{DC Motor Model}
When analyzing how to detect a stalled-rotor condition, it is useful to understand the first-principles electrical model of a DC motor, shown in Figure \ref{fig:dc-motor-model}.
\begin{figure}[H]
    \centering
    \begin{circuitikz}[american]
        \draw
        (0,0) to[short, o-] (1,0)
        to[R, l=$R$] (3,0)
        to[L, l=$L$] (6,0)
        to[vsource, l_=$V_{emf}$] (6,-3)
        to[short] (0,-3)
        to[vsource, l_=$V_{in}$, invert] (0,0)
        ;
    \end{circuitikz}
    \caption{Electrical Model of a DC Motor with Upright Voltage Sources}
    \label{fig:dc-motor-model}
\end{figure}

\newnoindentpara Where $V_{in}$ is the input voltage to the motor, $R$ is the resistance of the motor windings, $L$ is the inductance of the motor windings, and $V_{emf}$ is the back electromotive force (EMF) generated by the motor as it spins - $V_{emf} = K_e \cdot \omega$, where $K_e$ is the motor's back EMF constant and $\omega$ is the angular velocity of the motor shaft. Additionally, the torque generated by the motor is given by $T = K_t \cdot I$, where $K_t$ is the motor's torque constant and $I$ is the current flowing through the motor windings.\footnote{Note that for a given motor, $K_e$ and $K_t$ are numerically equal when using SI units.}

\subsection{Fundamental Principle of Stalled-Rotor Detection}
As we've defined a stalled-rotor condition to be one where the motor's rotor is not rotating ($\omega = 0$), we can leverage the DC motor model to inform our detection strategy. Since $V_{emf} = K_e \cdot \omega$, when the rotor is stalled, $V_{emf}$ will be zero. By extension, the motor's current draw will be at its maximum - all of the input voltage $V_{in}$ will be dropped across the motor's resistance and inductance, leading to a high current draw according to Ohm's Law ($I = \frac{V_{in}}{R}$ for a stalled motor). A schematic of this concept is shown in Figure \ref{fig:dc-motor-stall}.

\begin{figure}[H]
    \centering
    \begin{circuitikz}[american]
        \draw
        (0,0) to[short, o-] (1,0)
        to[R, l=$R$] (3,0)
        to[L, l=$L$] (6,0)
        to[short] (6,-3)
        to[short] (0,-3)
        to[vsource, l_=$V_{in}$, invert] (0,0)
        ;
        \draw[->, thick] (2.5, 0.5) -- (3.5, 0.5) node[midway, above] {$I$};
    \end{circuitikz}
    \caption{DC Motor Electrical Model with Shunted Back-EMF and Current Direction}
    \label{fig:dc-motor-stall}
\end{figure}

\newnoindentpara As a result, we can detect a stalled-rotor condition either by monitoring the motor's current draw (as a high current draw indicates a stall) or by monitoring the motor's angular velocity (as a zero angular velocity indicates a stall). 



\end{document}