\documentclass[main.tex]{subfiles}
\begin{document}

\section{How would you sense the voltage of a battery with a microcontroller?} \label{sec:voltage_divider} 
% Question is deliberately vague and short. The question is often asked in this manner in real interviews. 
% The term "ADC" is not given in the question to test if an interviewer is familiar with the concept and also to bait interviewees into given excessively complex nonsensical solutions. 
% An example of an excessively complex non-senical solution an interviewee could propose is using an ADC that supports really high input voltages such as ADS125H01 https://www.ti.com/product/ADS125H01 . The question is meant to leave bait for something like that and the first paragraph is intended to dispel this pitfall. 
The nominal voltage of the battery is 24 volts. Consider how your solution behaves when the battery's voltage varies. 
% This clarification to nominal voltage is given to ensure interviewees understand the battery voltage is higher than a standard microcontroller's ADC maximum voltage without giving away too much of the solution while also being low enough to avoid high voltage considerations in the solution.
% A maximum input voltage to microcontroller ADC is not given deliberately. Interviewees are expected to infer as a vast majority real microcontroller in the world don't feature a 24 V ADC. This is done to align with how the question is asked in real interviews. The request for extra consideration of extrema is done to motivate the "Pin Overvoltage" section. 

\spoilerline

\subsection{Implementation}
\noindent At first glance, an analog-to-digital converter (\textit{ADC}) could be used to directly sample the battery voltage. While this is a valid solution, it is not a standard practice as high voltage ADCs are less commonly included in microcontrollers and are more expensive to implement discretely. For this reason, the use of a low voltage ADC commonly found in microcontrollers is preferred. \newline

\newnoindentpara To do this, a voltage divider circuit can be used to scale the voltage of the battery by a fixed ratio, specified by the two resistors, such that the ADC can sample the signal without exceeding the maximum input voltage threshold.

\begin{figure}[H]
    \begin{center}
        \begin{circuitikz}[american]
            \draw (-0.5, 3.5) node[anchor=east] {$V_{in}$} -- (0, 3.5) -- (0, 3); 
            \draw (0, 3) to[resistor, l=$R_t$] (0, 1.5) to[resistor, l=$R_b$] (0, 0);
            \draw (0, 0) node[ground]{};
            \draw (0, 1.5) -- (0.75, 1.5) node[right] {$V_{out}$};
        \end{circuitikz}
        % Do not rotate voltage divider, resistors are vertical deliberately for common drawing method intuition.
        \caption{Voltage Divider Circuit}
        \label{fig:voltage_divider}
    \end{center}
\end{figure}

\subsection{Motivation}
Electronic systems often require high power actuators and batteries to operate - relatively high battery bus voltages are often selected by engineers to minimize current and conduction losses. Microcontrollers are often designed with low input voltages (i.e. 1.8V, 3.3V, 5V) in order to reduce power consumption and transistor size during operation. As chemical batteries charge and discharge the battery, the battery's voltage varies due to numerous factors (notably, state of charge). As a result, many battery-based embedded systems feature a battery-voltage sensing circuit hooked into a microcontroller to monitor the battery's voltage, and to take action when the battery voltage is too low or too high.

\subsection{Voltage Divider Circuit}
Ohm's Law relates the voltage across a resistor, $V$, to current flowing through the resistor, $I$, by $V = I \cdot R$. Applying Ohm's Law to each of the two resistors shown results in Equations \eqref{eq:ohms_law_top} and \eqref{eq:ohms_law_bot}. Because the resistors are in series, the current, $I$, flowing through them is the same ($I = I_{R_t} = I_{R_b}$).
\begin{equation}
    \begin{aligned}[b]
        V_{in} - V_{out} &= I \cdot R_t \\
        I &= \frac{V_{in} - V_{out}}{R_t}
    \end{aligned}
    \label{eq:ohms_law_top}
\end{equation}

\begin{equation}
    \begin{aligned}
        V_{out} - 0 &= I \cdot R_b \\
        I &= \frac{V_{out}}{R_b}
    \end{aligned}
    \label{eq:ohms_law_bot}
\end{equation}

\noindent Solving the system of equations to eliminate $I$ gives Equation \eqref{eq:voltage_divider} as follows. 

\begin{equation}
    \begin{aligned}[b]
        \frac{V_{out}}{R_b} &= \frac{V_{in} - V_{out}}{R_t} \\
        V_{out} \cdot R_t &= R_b \cdot (V_{in} - V_{out}) \\
        V_{out} \cdot R_t &= V_{in} \cdot R_b - V_{out} \cdot R_b \\
        V_{out} \cdot R_t + V_{out} \cdot R_b &= V_{in} \cdot R_b \\
        V_{out} \cdot (R_t + R_b) &= V_{in} \cdot R_b \\
        \frac{V_{out}}{V_{in}} &= \frac{R_b}{R_t + R_b}
    \end{aligned}
    \label{eq:voltage_divider}
\end{equation}

\noindent This indicates, based on the values chosen for $R_t$ and $R_b$, $V_{out}$ is scaled down from $V_{in}$ by a ratio determined by $R_t$ and $R_b$. Given a maximum ADC voltage rating, $V_{out}$, and a maximum battery voltage, $V_{in}$, a desired ratio between $R_t$ and $R_b$ can be determined. \newline

\newnoindentpara However, there are additional considerations in selecting $R_t$ and $R_b$, as the voltage divider circuit draws current proportional to the sum of its resistance ($R_{sum} = R_{t} + R_{b}$). Electrical power dissipation is given by $P = V \cdot I$. Substituting in Ohm's Law to describe the power dissipation of a resistor gives $P = \frac{V^2}{R}$ and $P = I^{2} \cdot R$. Note that if the input voltage of the battery is roughly fixed at some nominal $V_{in}$, then the circuit's power dissipation increases as $R_{sum}$ decreases. This power dissipation (\textit{referred to as quiescent current}) occurs constantly as voltage is always supplied to the circuit and can be significantly wasteful. \newline
% Introducing power dissipation formula in this question is important in my opinion as it opens intuition for the future.

\newnoindentpara If the $R_{sum}$ is too large, then the current flowing in the resistor divider is so small that noise coupling into the signal or input bias current into the ADC can result in significant measuring error.\footnote{A common value selected for $I$ is roughly 1mA of current to flow in a sensing resistor divider, though different values may be seen based on the application.}

\subsection{Voltage Divider Intuition}
Consider a few specific cases of this circuit to help build intuition to approach problems featuring the voltage divider.
\begin{itemize}
    \item A simple case of the voltage divider circuit is when both resistors have the same value. $R = R_t = R_b$. In this case, $\frac{V_{out}}{V_{in}} = \frac{R}{R+R} = \frac{R}{2 \cdot R} = \frac{1}{2}$ which means $V_{in} = 2 \cdot V_{out}$ or $V_{out} = \frac{1}{2} \cdot V_{in}$.
    \item Because $R_t > 0 \ \Omega$ and $R_b > 0 \ \Omega$ are required (as negative resistors do not exist), in all cases of the circuit being employed we observe that $0 < \frac{V_{out}}{V_{in}} < 1$. This indicates that $V_{out}$ < $V_{in}$ always holds for the voltage divider so the circuit always scales a voltage down from its input to its output.
    \item This circuit assumes no source impedance from $V_{in}$ and no loading connected to $V_{out}$. However, this assumption is not always valid in practical circuits (and explored later in this answer).
\end{itemize}

\subsection{Input Bias Current}
When an ADC is operating nominally, it suffers from a non-ideality known as input bias current.\footnote{Note that input bias current may also be represented by input impedance in which the load current is instead modelled by a loading resistance to ground, however, the concept remains similar.} This input bias current, $I_{bias}$, is in the micro-amp range, is used to feed the internal analog circuit, and is present to varying degrees of severity in all forms of ADCs. This is a problem because it adds loading to our resistor divider and results in Equation \eqref{eq:voltage_divider} being incorrect. 

\begin{figure}[H]
    \begin{center}
        \begin{circuitikz}[american]
            \draw (-0.5, 3.5) node[anchor=east] {$V_{in}$} -- (0, 3.5); 
            \draw (0, 3.5) to[resistor, l=$R_t$, i=$I_{R_t}$] (0, 1.5);
            \draw (0, 1.5) to[resistor, l=$R_b$, i=$I_{R_b}$] (0, 0);
            \draw (0, 0) node[ground]{};
            \draw (0, 1.5) -- (2.5, 1.5) node[right] {$V_{out}$};
            \draw (1.5, 1.5) to[isource, l=$I_{bias}$] (1.5, 0);
            \draw (1.5, 0) node[ground]{};
        \end{circuitikz}
        \caption{Voltage Divider Circuit with Loading}
        \label{fig:voltage_divider_loaded}
    \end{center}
\end{figure}

\begin{equation}
    \begin{aligned}[b]
        % Commented out on purpose I just want to be sure I didn't mess up the solution
        % I_{R_t} &= I_{R_b} + I_{load} \\
        % I_{R_b} &= \frac{V_{out}}{R_b} \\
        % I_{R_t} &= \frac{V_{in} - V_{out}}{R_t} \\
        % \frac{V_{out}}{R_b} + I_{load} &= \frac{V_{in} - V_{out}}{R_t} \\
        % V_{out} \cdot R_t + I_{load} \cdot R_t \cdot R_b &= R_b \cdot (V_{in} - V_{out}) \\
        % V_{out} \cdot R_t + I_{load} \cdot R_t \cdot R_b &= R_b \cdot V_{in} - R_b \cdot V_{out} \\
        % V_{out} \cdot R_t + R_b \cdot V_{out} &= R_b \cdot V_{in} - I_{load} \cdot R_t \cdot R_b \\
        % V_{out} \cdot (R_t + R_b) &= V_{in} \cdot (R_b - \frac{I_{load} \cdot R_t \cdot R_b}{V_{in}}) \\
        % \frac{V_{out}}{V_{in}} &= \frac{R_b - \frac{I_{load} \cdot R_t \cdot R_b}{V_{in}}}{R_t + R_b} \\
        % \frac{V_{out}}{V_{in}} &= \frac{R_b}{R_t + R_b} - \frac{I_{load} \cdot R_t \cdot R_b}{(R_t + R_b) \cdot V_{in}} \\
        \frac{V_{out}}{V_{in}} &= \frac{R_b}{R_t + R_b} - \frac{I_{load}}{V_{in}} \cdot \frac{R_t \cdot R_b}{R_t + R_b}
    \end{aligned}
    \label{eq:loaded_voltage_divider}
\end{equation}

\noindent Using the circuit in Figure \ref{fig:voltage_divider_loaded} results in an adjusted transfer function shown in Figure \ref{eq:loaded_voltage_divider}. Input bias current is also hard to model and can vary significantly so the simplest method of compensating for this by reducing $R_{sum} = R_t + R_b$ which increases $R_t || R_b = \frac{R_t \cdot R_b}{R_t + R_b}$ and decreases the entire second term of Figure \ref{eq:loaded_voltage_divider}. \newline

% Consider converting this into a follow-up
\newnoindentpara Another method to compensate for a large input bias current is to use an external voltage buffer, aka a unity gain op-amp (\textit{Operational Amplifier}) to repeat the voltage, but buffer the current. 

\begin{figure}[H]
    \begin{center}
        \begin{circuitikz}[american]
            \draw (-0.5, 3.5) node[anchor=east] {$V_{in}$} -- (0, 3.5) -- (0, 3); 
            \draw (0, 3) to[resistor, l=$R_t$] (0, 1.5) to[resistor, l=$R_b$] (0, 0);
            \draw (0, 0) node[ground]{};
            \draw (0, 1.5) -- (0.75, 1.5);
            \draw (0.75,1.5) node[left]{} to[short] ++ (1,0)
                node[op amp, noinv input up, anchor=+](opamp){}
                (opamp.-) -- ++(0,-1) coordinate(FB)
                (FB) to[short] (FB -| opamp.out) -- (opamp.out)
                to [short] ++(1,0) node[right]{$V_{out}$};
            \label{fig:bufferred_divider}
        \end{circuitikz}
        \caption{Op-amp Buffered Resistor Divider}
    \end{center}
\end{figure}

\noindent Op-amps also have input bias current, however, this can be compensated for it using the circuit in Figure \ref{fig:bufferred_divider_comp} where $R_c = R_t || R_b$.

\begin{figure}[H]
    \begin{center}
        \begin{circuitikz}[american]
            \draw (-0.5, 3.5) node[anchor=east] {$V_{in}$} -- (0, 3.5) -- (0, 3); 
            \draw (0, 3) to[resistor, l=$R_t$] (0, 1.5) to[resistor, l=$R_b$] (0, 0);
            \draw (0, 0) node[ground]{};
            \draw (0, 1.5) -- (0.75, 1.5);
            \draw (0.75,1.5) node[left]{} to[short] ++ (1,0)
                node[op amp, noinv input up, anchor=+](opamp){}
                (opamp.-) -- ++(0,-1) coordinate(FB)
                (FB) to[R=$R_c$] (FB -| opamp.out) -- (opamp.out)
                to [short] ++(1,0) node[right]{$V_{out}$};
        \end{circuitikz}
        \caption{Compensated Op-amp Buffered Resistor
         Divider}
         \label{fig:bufferred_divider_comp}
    \end{center}
\end{figure}


\subsection{Pin Overvoltage}
% Leakage explanation offers an introduction to input bias current. Is important to consider in a real solution as battery voltage is likely to spike/overvolt in embedded systems with high power actuators. 
Microcontroller pins often feature clamping diodes to protect the device from some transient voltages outside of the permissible operating range. ESD (\textit{Electro-static discharge}) is an example of a potentially destructive transient event. The use of external clamping diodes is common to protect for higher power transients in addition to internal clamping. \newline

\begin{figure}[H]
    \begin{center}
        \begin{circuitikz}[american]
            \draw (0, 3) node[vcc]{}; 
            \draw (0, 1.5) to[empty diode] (0, 3);
            \draw (-2, 1.5) -- (0.5, 1.5);
            \draw[dashed] (0.5, 1.5) -- (1.5, 1.5);
            \draw (0, 0) to[empty diode] (0, 1.5);
            \draw (0, 0) node[ground]{};
            \draw (-2, 1.5) node[anchor=east] {IO Pin};
            \draw[thick] (-1, 4.5) rectangle (3, -1);
            \draw (1, 4) node[]{Microcontroller};
        \end{circuitikz}
        \caption{Microcontroller Pin with Clamping Diodes}
        \label{fig:clamping_diodes}
    \end{center}
\end{figure}

\noindent These clamping diodes, shown in Figure \ref{fig:clamping_diodes}, will conduct current when a voltage is applied that exceeds the microcontroller's supply voltage, $V_{cc}$, and when a voltage is applied that is below the microcontroller's ground reference, $V_{ss}$. They will conduct current unless if the current becomes excessive resulting in damage to the diodes and consequently damage to the device. Allowing the microcontroller clamping diodes to sink some current during an overvoltage event is permissible. Note that when an ADC pin is overvolted, accurate ADC readings cannot be expected. \newline

\newnoindentpara External TVS diodes are also used when faster response times are required. A disadvantage of external protection diodes is that they consume some leakage current which will result in less accurate ADC measurements. This leakage current is often difficult to model (non-linear) and can be dependent on numerous factors.

\subsection{Analog to Digital}
An ADC (\textit{Analog to Digital Converter}) is a component that, as the name implies, converts an analog value into a digital value. The frequency components of the sampled signal and sample rate are critical design decisions to avoid aliasing. Aliasing is a large concept in sampling theory that will not be explored in this guide, however, Tim Wescott's article titled \bluehref{https://www.wescottdesign.com/articles/Sampling/sampling.pdf}{Sampling: What Nyquist Didn’t Say, and What to Do About It} provides a deeper understanding of this sampling theory. \newline
% Paragraph provides motivation for why we want a low pass capacitor. I don't want to do too much detail, I think if you can say the word "Aliasing" in an interview it's plenty.

\newnoindentpara To avoid aliasing, it is common to see a low pass filter (LPF) added to voltage divider circuits (a capacitor added in parallel to the $V_{out}$ signal) to filter out higher frequency noise. The cutoff frequency of a LPF is the frequency in which the circuit will attenuate to half its input power or $\frac{1}{\sqrt{2}}$ of its input voltage. The cutoff frequency of this low pass filter is usually selected to be roughly five times lower than the sampling frequency to avoid aliasing\footnote{Refer to LPF theory as to why the cutoff frequency is chosen to be higher - the need arises due to the cutoff being -20dB/decade}. 
% Just intended to bring up the concept and build intuition, no proof or definitions or math is worth giving here. 

\begin{figure}[H]
    \begin{center}
        \begin{circuitikz}[american]
            \draw (-0.5, 3.5) node[anchor=east] {$V_{in}$} -- (0, 3.5) -- (0, 3); 
            \draw (0, 3) to[resistor, l=$R_t$] (0, 1.5) to[resistor, l=$R_b$] (0, 0);
            \draw (0, 0) node[ground]{};
            \draw (0, 1.5) -- (2.5, 1.5) node[right] {$V_{out}$};
            \draw (1.5, 1.5) to[capacitor, l=$C$] (1.5, 0);
            \draw (1.5, 0) node[ground]{};
            \label{fig:voltage_divider_low_passed}
        \end{circuitikz}
        \caption{Voltage Divider Circuit with Anti-Aliasing Capacitor}
    \end{center}
\end{figure}

\subsection{Layout Considerations}
When placing a voltage divider circuit on a PCB, consider:
\begin{itemize}
    \item The $V_{out}$ trace should be as short as possible to avoid noise from coupling into the signal. 
    \item The low pass capacitor should be placed near the ADC pin so it can filter out noise that couples into $V_{out}$ before the ADC samples it.
\end{itemize}

\subsection{Follow-ups}
\begin{itemize}
    \item Reducing the quiescent current of a voltage divider can be done by increasing $R_{sum}$, however, this only gets you so far. Propose a simple circuit to disable current consumption when the voltage divider is not needed.\footnote{A solution is given in extra practice question \ref{extra_practice:switched_divider}}
    \item What if all analog input pins of the microcontroller are already in use? % analog mux, extremely cheap 
\end{itemize}

\end{document}
